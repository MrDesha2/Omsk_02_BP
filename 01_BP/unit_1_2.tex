
\subsection{Описание бизнес-процесса «Проектирование и разработка новой продукции»}

\subsubsection{Сценарий ''Получение запроса на разработку новой продукции''}
\label{bp:pm_1}

\begin{enumerate}

\item \manager получает от заказчика требования в виде чертежей, готовой ТК, образца короба, продукции или требования с указанием размеров продукции, веса вложения, прочностных характеристик и т.д.
\item \manager заполняет запрос на разработку чертежа конструкции/дизайна графики (процесс \textbf{«Продажа готовой продукции»}) в системе \erp в случае возникновения следующих потребностей со стороны клиента:
    
    \begin{enumerate}
        \item разработка нового вида продукции;
        \item внесение изменений в уже существующие чертежи конструкции/ дизайн графики;
\item изменение размера, удаление или добавление новых элементов;
\item изменение профиля;
\item при необходимости изготовления образца(ов) коробов по готовому чертежу конструкции.
    \end{enumerate}


\item	Для оформления запроса \manager вручную создает в системе \erp новую характеристику номенклатуры \myobject{Характеристика}, при этом будет создан документ \myobject{ТехнологическаяКарта}.
\item \manager сохраняет документ. Система \erp автоматически создает в системе \gofro документ \myobject{ЗаявкаСпецификация}.
Обратно системе \gofro вернет ссылку на созданный документ, который будет прописан в системе \erp в документе \myobject{Технологическая карта}.

\item Для заполнения дополнительных требований по изделию \manager по созданной ссылке переходит в системе \gofro и заполняет параметры в документе \myobject{ЗаявкаСпецификация}
\item	При наличии файлов от клиента \manager в системе \gofro прилагает их к документу \myobject{ЗаявкаСпецификация} на вкладке «Файлы».


\item \tehnolog просматривая обновления журнала документов \myobject{ЗаявкаСпецификация} обрабатывает требование на разработку ТК в системе \gofro документы со статусом ''Новый''.

% В случае разработки нестандартной продукции \tehnolog согласовывает возможность производства изделия с \processengineer.

\item \tehnolog создает предварительный чертеж (если требуется выпуск образцов) или рабочий чертеж в программе \blue{\@AutoCad}.
\item	При необходимости \tehnolog в системе \gofro должен создать элемент справочника \myobject{ТехнологическаяКарта}, в которую будут автоматически скопированы поля из документа \myobject{ЗаявкаСпецификация}, которые заполнил \manager. Правило нумерации: код техкарты – сквозная нумерация. Номер сквозной с префиксом "ГТ. . . "


\item \tehnolog 
% переводит чертежи и дизайн в формат «dxf» и 
отправляет чертеж по электронной почте изготовителю оснастки (сторонняя компания) и самостоятельно размещает заказ на изготовление штанцформы согласно процесса \textbf{«Заказ штанцевальных форм/клише и входной контроль качества»}. 

\item \tehnolog подгружает чертеж в систему \gofro в \myobject{ТехнологическаяКарта}.
%\item	\manager заказывает разработку дизайна штанцформы у сторонней компании согласно процесса \textbf{«Заказ штанцевальных форм/клише и входной контроль качества»}. 

\item \tehnolog «привязывает» полученный файл с разверткой штанцформы в соответствующих полях справочника \myobject{ТехнологическаяКарта}.

\item \manager при необходимости в системе \gofro размещает заказ на изготовление печатной формы согласно процесса \textbf{«Заказ штанцевальных форм/клише и входной контроль качества»}. 


% \item \tehnolog разрабатывает дизайн на основании требований \blue{\#ЗаявкаСпецификация} в системе \blue{\@CorelDraw}.
\item \tehnolog подгружает макет печати в систему \gofro в \myobject{ТехнологическаяКарта}.

\item	\manager заказывает дизайн печатной формы у компании поставщика согласно процесса \textbf{«Заказ штанцевальных форм/клише и входной контроль качества»} (подробнее в пункте \ref{bp:toolrequest}). Полученный макет \manager пересылает \tehnolog по почте или добавлением в документ \myobject{ЗаявкаСпецификация}.
\tehnolog «привязывает» полученный файл с разверткой печатной формы в соответствующих полях справочника \myobject{ТехнологическаяКарта}. 

%\todo{РОМА. прописать заказ печатных форм} ВЫПОЛНЕНО 

% \item	Для стандартных изделий без печати \tehnolog выполняет расчет развертки в системе \gofro.


\item	\tehnolog  создает при необходимости новый элемент справочника \myobject{Оснастка} (штанц-форма, флексо-форма), указывает номенклатуру оснастки из справочника \myobject{Номенклатура}. 
\item	\tehnolog  указывает оснастку (штанц-форму) в соответствующем элементе справочника \myobject{ТехнологическаяКарта}. 
\item \tehnolog  указывает оснастку (печатную форму) в соответствующем элементе справочника \myobject{ТехнологическаяКарта}.
\item	После ввода данных по конструкции и графике изделия в созданной в системе \gofro технологической карте \tehnolog заполняет остальную информацию в справочнике \myobject{ТехнологическаяКарта}.
% \item	\designer в системе \gofro заполняет заготовку для изделия и технологические маршруты для изготовления изделия в справочнике \#ТехнологическаяКарта. В случае, если изделие выпускается из покупной заготовки, \designer все равно указывает в маршруте на первом шаге гофроагрегат (но он не планируется для него).
\item	\tehnolog  вводит в систему \gofro информацию по технологическим маршрутам в заготовке  \myobject{ТехнологическаяКарта} с указанием (при необходимости) признака использования оснастки на соответствующем шаге маршрута. При этом в \gofro имеется механизм по автоматическому созданию маршрутов по шаблонам, заложенным в настройках системы.

\item	\tehnolog выполняет детальную проверку и корректировку возможного маршрута.
\item	\tehnolog указывает требования к упаковке и транспортировке изделия в \myobject{ТехнологическаяКарте} на основании требованийЮ, указанных в документе \myobject{ЗаявкаСпецификация}.
\item	\tehnolog указывает в системе \gofro в справочнике \myobject{Номенклатура} ссылку на справочник \myobject{ТехнологическаяКарта}.
\item	После разработки технологической карты \tehnolog меняет статус в документе \myobject{ЗаявкаСпецификация} на “Выполнен”. 
\item	\manager в системе \erp открывает форму созданного элемента \myobject{ТехнологическаяКарта} из документа \myobject{ТехнологическаяКарта} по ссылке. При этом будет открыта форма объекта \myobject{ТехнологическаяКарта} в системе \gofro.
\manager вызывает по команде \myform{ПечататьТК} печатную форму элемента \myobject{ТехнологическаяКарта} и отправляет из системы \gofro форму в формате pdf или распечатывает, подписывает печатную форму ТК у клиента на бумажном носителе с указанием даты подписания и фамилий лиц с последующей передачей в архив и размещением подписанной клиентом отсканированной копии документа в системе \gofro в созданном элементе \myobject{ТехнологическаяКарта}.

\item	Технологические карты изделий, выпуск продукции по которым больше не производится,  \tehnolog в системе \gofro переводит по запросу от \manager в статус «Архивные», при этом создать новый производственный заказ на такое изделие больше невозможно.

\end{enumerate}





\subsubsection{Сценарий ''Валидация проекта (Утверждение чертежей у клиента)''}
\label{bp:pm_2}

\begin{enumerate}
\item \manager в системе \erp находит необходимую номенклатуру готовой продукции, выбирает характеристику номенклатуры. В поле ''ТехнологическаяКарта'' будет указана ссылка на ТК в системе \gofro.
\item \manager  в системе  \gofro открывает форму созданного элемента \myobject{ТехнологическаяКарта}, вызывает по команде \blue{\$ПечататьТК} печатную форму элемента \myobject{ТехнологическаяКарта} (отчет «Отчет для клиента») и отправляет клиенту для дальнейшего обязательного утверждения из системы \gofro форму в формате pdf или распечатывает, подписывает печатную форму ТК у клиента на бумажном носителе с указанием даты подписания и фамилий лиц с последующей передачей в архив и размещением подписанной отсканированной копии документа в системе \gofro в созданном элементе \myobject{ТехнологическаяКарта}.
 \item	После согласования технологической карты клиентом \manager  пересылает сканированную версию ТК  \tehnolog. 
 \item \tehnolog должен в системе \gofro в справочнике \myobject{ТехнологическаяКарта} прикрепить отсканированные версии подписанных клиентом чертежа и графики.
\item	После проверки данных и заполнения маршрута \tehnolog  
% с выделенными правами 
в элементе справочника
\myobject{ТехнологическаяКарта} в системе \gofro устанавливает статус «Активна» и дальнейшие изменения в \myobject{ТехнологическаяКарта} становятся недоступны другим пользователям.

\end{enumerate}




% \subsubsection{Сценарий ''Производство образцов продукции''}
% Выполняется как обычный заказ

% \label{bp:pm_3}

% \begin{enumerate}
% \item \manager  в системе  \gofro о создает запрос на разработку образцов продукции (процесс «Продажа готовой продукции») в случае возникновения следующих потребностей со стороны клиента:

% \begin{enumerate}
% \item 	разработка нового вида продукции;
% \item  при необходимости изготовления образца(ов) коробов по готовому чертежу конструкции.


% \end{enumerate}
% \item	Для оформления запроса \manager вручную создает  в системе  \gofro документ  \blue{\#ЗаявкаНаОбразцы}.
% \item	При наличии файлов от клиента \manager в системе \gofro прилагает их к документу \blue{\#ЗаявкаНаОбразцы} на вкладке «Файлы».
% \item	\manager указывает желаемый срок изготовления образцов и сохраняет документ \blue{\#ЗаявкаНаОбразцы}.
% \item	\tehnolog просматривает журнал документов  \blue{\#ЗаявкаНаОбразцы} в системе  \gofro и выбирает для работы проведенный документ со статусом «Новый».
% \item	\tehnolog изготавливает образцы согласно полученной заявке.
% \item	По факту готовности образцов \tehnolog в системе \gofroуказывает в документе \blue{\#ЗаявкаНаОбразцы} статус «Выполнен».

% \end{enumerate}



\subsubsection{Сценарий ''Контроль за изменениями в технологических картах''}
\label{bp:pm_4}


\begin{enumerate}

\item	При необходимости изменения активного элемента \myobject{ТехнологическаяКарта}  (например, изменение печати) возможны два варианта:

\begin{enumerate}
\item	Пользователь с правом «Перевод ТК в Разработку» может поменять статус \myobject{ТехнологическаяКарта} на «В разработке» и остальные пользователи снова смогут ее редактировать.
\item	Пользователь с выделенным правом «Можно менять активные ТК» может изменить поля в техкарте со статусом «Активна»
\end{enumerate}
Новый документ \myobject{ТехнологическаяКарта} при этом не создается.
\end{enumerate}


% \subsubsection{Синхронизация с КИС}
% \label{bp:pm_integration}

% \begin{enumerate}
% \item Выгрузка справочных данных из \erp в \gofro производится автоматически.
% \begin{enumerate}
% \item 	Справочник  \#Номенклатура по оснастке из \gofro загрузится в Справочник  \#Номенклатура системы  \erp.

% \end{enumerate}

% \end{enumerate}