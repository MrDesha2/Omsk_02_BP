\newpage
\subsection{Описание бизнес-процесса «Учет ТМЦ»}
\label{bp:storage}

\subsubsection{Сценарий ''Прибытие автомобиля на склад''}
\label{bp:storage_1}


\begin{enumerate}

\item	\kladovshik получает от водителя приходные документы от поставщика.

\end{enumerate}




\subsubsection{Сценарий ''Предварительный контроль качества сырья''}
\label{bp:storage_2}


\begin{enumerate}

\item  Водитель погрузчика производит выгрузку сырья, \kladovshik осматривает его визуально. При обнаружении дефектов сообщает кладовщику.

\item	\kladovshik 
% в присутствии \okk 
осматривает ТМЦ на соответствие заявленным характеристикам, сортности, целостности упаковки. При наличии отклонений составляет акт несоответствия.

\item Забракованное сырье водитель погрузчика должен отставить на склад брака.

\item \kladovshik при выявлении нарушения транспортировки составляет акт о повреждении и передает акт в бухгалтерию. Рулон принимается на склад в любом случае.
\end{enumerate}


\subsubsection{Сценарий ''Ввод новых элементов справочника Номенклатура''}
\label{bp:storage_4}



\begin{enumerate}
\item При появлении новых позиций ТМЦ \kladovshik  добавляет новые записи в справочник \myobject{Номенклатура} в системе \erp. Справочник выгружаются по регламенту в систему \gofro 
% и \erp 
% в справочник \myobject{Номенклатура} в систему \gofro.
\item Новые позиции по готовой продукции (гофропроизводство) добавляет \manager в системе \erp. Справочник выгружается по регламенту из системы \erp в справочник \myobject{Номенклатура} в систему \gofro.
\end{enumerate}



\subsubsection{Сценарий ''Порулонная приемка сырья''}
\label{bp:storage_5}

\begin{enumerate}
\item \kladovshik контролирует вес и номер рулона, после чего сравнивает с весом по номеру рулона в отвес-фактуре поставщика.

\item \kladovshik создает в системе \erp документ ”Приходный ордер” (\myobject{Приходный ордер})
% , в этом же документе распечатывает этикетки для каждого рулона 
и регистрирует поступление рулонов.
Регистрация поступления рулонов выполняется по каждому рулону с присвоением номера рулона в виде QR-кода.
\kladovshik по каждому рулону  вводит серию с указанием веса рулона, номера рулона. 

\item Контроль за остатками материалов осуществляется в системе \erp.

\item \kladovshik проводит документ \myobject{Приходный ордер}.
\item \kladovshik передает сопроводительные документы по поступлению сырья в бухгалтерию.
% \item \kladovshik на основе приходных документов от поставщика в системе  \erp находит документ #ПоступлениеТМЦ, заполняет цены на сырье из сопроводительных документов и проводит документ.


\end{enumerate}


\subsubsection{Сценарий ''Поступление полуфабрикатов (покупных заготовок) и материалов, используемых на ГА (крахмала, едкого натра, буры)'' }
\label{bp:storage_51}

\begin{enumerate}
\item Первичный учет поступления ведется в системе \erp.
\item \kladovshik контролирует количество и номенклатуру полуфабрикатов, сравнивая с данными документа поставщика.
\item \kladovshik  при поступлении в системе \erp создает документ \myobject{Приходный ордер}, при этом указывает поставщика и склад-получатель. 
\item \kladovshik  по каждой продукции системе \erp указывает позицию из справочника \myobject{Номенклатура}, количество и цену.
\item \kladovshik  проводит документ \myobject{Приходный ордер}.
\item \kladovshik передает сопроводительные документы по поступлению сырья в бухгалтерию.



\end{enumerate}



\subsubsection{Сценарий ''Перемещение рулонов''}
\label{bp:storage_6}

\begin{enumerate}
\item  	Первичный учет перемещения рулонов ведется в системе \erp.
\item 	\planner формирует план потребности по сырью на смену и передает \gaoperator.
% \item \kladovshik (\gaoperator) отслеживает в системе \gofro плановое время и количество подачи сырья \todo{Нет такого???}
%\item	\gaoperator в системе \stock печатает отчет <<ОтчетПоНеобходимомуСырью>> с разбивкой по времени и передает на склад \kladovshik.
% \item	\kladovshik  при перемещении рулонов в системе \erp создает документ \myobject{Перемещение},
% % \todo{Проверить наличие такого документа???}, 
% при этом указывает склад-отправитель и склад-получатель. 
\kladovshik по заявке списывает сырье с помощью ТСД. При этом кладовщик срезает этикетку с штрих-кодом с рулона для повторного контроля в конце смены.
В системе \erp ТСД создает документ \myobject{''Перемещение товаров''}. В конце смены \kladovshik в системе \erp проверяет созданный документ со срезанными ярлыками с штрих-кодом и проводит документ.
% \item 	\kladovshik  указывает номера рулонов и позиции \#Номенклатура для перемещения в документе \myobject{ПеремещениеТМЦ}.
\item 	\kladovshik проводит документ \myobject{Перемещение товаров}.
\end{enumerate}



\subsubsection{Сценарий ''Перемещение ТМЦ''}
\label{bp:storage_61}

\begin{enumerate}
\item  	Первичный учет перемещения прочих материалов ведется в системе \erp документом \myobject{Перемещение товаров}
% \item 	\planner формирует план потребности по сырью на смену и передает !МашинистуГА.
% \item	\gaoperator в системе \gofro печатает отчет ОтчетПоНеобходимомуСырью с разбивкой по времени и передает на склад \kladovshik.
\item	\kladovshik при перемещении ТМЦ  в системе \erp создает документ \myobject{Перемещение товаров}, при этом указывает склад-отправитель и склад-получатель.
\item 	\kladovshik указывает позиции \myobject{Номенклатура} для перемещения в документе \myobject{Перемещение товаров} и количество.
\item 	\kladovshik проводит документ \myobject{Перемещение товаров} .
\end{enumerate}



% \subsubsection{Сценарий ''Списание рулонов бумаги и картона''}
% \label{bp:storage_7}

% \todo{Будем замарачиваться??? Им точно надо???}

% \begin{enumerate}
% \item  	Первичное списание рулонов бумаги и картона производится в системе \gofro на раскатах гофроагрегатов. 
% %\item  	Первичное списание рулонов бумаги и картона производится в системе \stock.
% %\item \gaoperator при учете сырья указывают его в бумажном бланке учета сырья.
% \item \gaoperator на раскате создает в системе \gofro документ \blue{\#СырьеДляВыработки}.
% \item \gaoperator на раскате в документе \blue{\#СырьеДляВыработки} 
% считывает сканером штрих-кода номер каждого рулона (на основании маркировочного ярлыка рулона) в колонку, соответствующую номеру слоя выпускаемой композиции (\textbf{Внимание! Номер слоя может не соответствовать номеру раската и вносить необходимо именно номер слоя}). После указания номера рулона система \gofro  автоматически заполняет вес рулона на основании складских остатков.
% \item \gaoperator на раскате при снятии рулона считывает номер рулона с ярлыка и указывает конечный диаметр (???) рулона в системе \gofro. Система \gofro рассчитывает расход сырья по рулону на основании разницы между начальным и конечным диаметрами рулона.
% % \todo[inline]{ВНИМАНИЕ!!! Оптисофт считает, что крайне нежелательно работать с диаметром. Необходимо взвешивать рулоны. При расчете по формуле постоянно будут происходить неверные вычисления оставшегося веса в рулоне. Если все-таки придется работать с диаметром, то Оптисофт ждет от Предприятия формулу по расчету веса на основании измеренного диаметра.}

% \item 	Если рулон был смотан полностью, \gaoperator на раскате в системе \gofro указывает нулевой конечный диаметр.

% \item 	Если при выпуске раскроев на гофроагрегате происходит замена слоя по сравнению с тем, что было указано по заданию, то \gaoperator должен указать в строке с раскроем в системе \gofro соответствующий новый слой.

% \item \gaoperator в конце смены в документе \blue{\#ВыработкаГофроагрегата} для табличной части сырья вызывает команду \blue{\$ЗаполнитьМатериалы},  система \gofro заполняет сырье с учетом номеров рулонов на основании данных списания материалов по документу \blue{\#СырьеДляВыработки}. При этом таблица материалов будет заполнена фактически распределенными рулонами на выпуск продукции с указанием номера заказа, номенклатуры, фактического веса рулона. 


% \item	Документ  \blue{\#ВыработкаГофроагрегата} с признаком «Проверено» автоматически выгружается в систему \stock  в документ \blue{\#Перемещение}. При этом табличная часть по фактически потребленным материалам будет заполнена на основании данных документа \blue{\#ВыработкаГофроагрегата} системs \gofro \todo{Требуется уточнение???}.

% % После установки \master в документе \#СписаниеТМЦ признака «Проверено» вносить изменения в документ сможет только пользователь, обладающий соответствующими правами.

% % \item \gaoperator в конце смены создает документ #СписаниеТМЦ на основании документа #ВыработкаГофроагрегата. При этом система @ГТ автоматически заполняет табличную часть документа #СписаниеТМЦ сырьем, указанным в документе #ВыработкаГофроагрегата. 
% % 8.	!Учетчик проводит документ #СписаниеТМЦ.

% \end{enumerate}



\subsubsection{Сценарий ''Списание рулонов бумаги и картона''}
\label{bp:storage_8}

\bigskip

% \begin{enumerate}
% \item \kladovshik создает документ \myobject{СписаниеТМЦ} в системе \gofro.
% \item \kladovshik указывает позиции справочника \myobject{Номенклатура} для списания.
% \item \kladovshik проводит документ \myobject{СписаниеТМЦ}.

% \end{enumerate}

\begin{enumerate}
\item Первичное списание рулонов бумаги и картона в производство со склада сырья производится в системе \syncro на раскатах гофроагрегате. 
\item \gaoperator на ГА открывает смену в системе \gofro, при этом автоматически создается документ  \myobject{СырьеДляВыработки}.
\item При установке каждого нового рулона \linkoperator 
на каждом раскате считывает штрих-код рулона сканером ШК (\red{ЛИНЕЙНЫЙ}) в системе \syncro.
\item Система \syncro запрашивает параметры рулона у системы \gofro.

% заносит в таблицу рулонов новую строку, указывает номер раската, считывает ШК рулона сканером ШК.  Система @ERP на ТСД автоматически заполняет текущий вес рулона на основании остатков.

\item При неполном использовании рулона  \linkoperator снимает рулон. Транспортная система \newkuani забирает рулон, взвешивает, печатает этикетку с номером и ШК рулона. Система \syncro хранит остаточный вес рулона.
% \item \linkoperator указывает Вес снятого рулона на бирке.
\item Система \newkuani хранит вес снятого рулона.
\item Оставшиеся неполные рулоны система \newkuani перемещает в зону хранения рулонов гофроагрегата.
% \item Оставшиеся неполные рулоны \linkoperator дает указание карщику на перемение на склад.
\item В конце смены \kladovshik принимает остаточные рулоны на склад через ТСД в системе \erp по документу \myobject{ВозвратСырья}.
На склад возвращаются рулоны, с которых использовано менее 50\%.
\item \gaoperator в конце смены в документе \myobject{ВыработкаГофроагрегата} в системе \gofro нажимает команду \myobject{ЗаполнитьМатериалы},   при этом система \gofro автоматически загружает использование сырья на гофроагрегате из системы \syncro, заполняет сырье с учетом номеров рулонов. При этом таблица материалов будет заполнена фактически использованными рулонами на выпуск продукции с указанием номера заказа, номенклатуры, фактического веса рулона. 
\item \gaoperator при отклонении в использовании сырья в документе \myobject{ВыработкаГофроагрегата} корректирует факт использования сырья по данным системы \syncro.


\end{enumerate}



% \subsubsection{Сценарий ''Списание ТМЦ, используемых на операциях переработки (полуфабрикаты сторонних производителей, краска и др.) на основании отчетов производства''}
% \label{bp:storage_9}


% \begin{enumerate}
% \item Первичное списание прочих материалов при необходимости производится в системе \erp. 
% % 2.	!МастерЦГТ в документе #ВыработкаПоПереработке  заполняет в таблице “Материалы” в конце смены дополнительные материалы, использованные при производстве готовой продукции по факту.
% % 3.	!МастерЦГТ в конце смены создает документ #СписаниеТМЦ на основании документа #ВыработкаПоПереработке. При этом система @ГТ автоматически заполняет табличную часть документа #СписаниеТМЦ позициями материалов, указанными в документе #ВыработкаПоПереработке. 


% \end{enumerate}


\subsubsection{Сценарий ''Списание ТМЦ, используемых на операциях переработки (полуфабрикаты сторонних производителей, краска и др.) на основании отчетов производства''}
\label{bp:storage_10}


\begin{enumerate}
\item Первичное списание прочих материалов при необходимости производится в системе \erp. 
\item \kladovshik списывает материалы в конце смены документом  \myobject{Перемещение товаров} 
% \todo{В бух нет такого документа. Уточнение, чем списывать и где ???}. 
% Передача материалов в кладовую выполняется в произвольном количестве, то есть это ненормируемые производственные затраты.
\end{enumerate}


\subsubsection{Сценарий ''Проведение инвентаризации''}
\label{bp:storage_11}


\begin{enumerate}
\item Проведение инвентаризации по готовой продукции производится в системе \erp. 
\item Проведение инвентаризации по сырью (бумага и картон) производится в системе \erp. \item	\kladovshik при инвентаризации сырья в системе \erp на ТСД создает документ \myobject{ИнвентаризацияТМЦ}. \kladovshik сканирует на ТСД рулоны на складе. 

% \item	На предприятии создается инвентаризационная комиссия.
\item	По факту инвентаризации рулонов \kladovshik корректирует при необходимости в системе \erp в документе \myobject{ИнвентаризацияТМЦ} фактическое количество ТМЦ.
\item	Инвентаризация других видов ТМЦ \kladovshik создает в системе \erp документ \myobject{ИнвентаризацияТМЦ}, указывает номенклатуру и фактическое количество ТМЦ.
\item	По факту отклонения \kladovshik создает в системе документы \myobject{ОприходованиеТМЦ}, \myobject{ПоступлениеТМЦ} и \myobject{СписаниеТМЦ}.

% \item \kladovshik списывает материалы в конце смены документом \#ПередачаМатериаловВКладовую. 
% Передача материалов в кладовую выполняется в произвольном количестве, то есть это ненормируемые производственные затраты
\end{enumerate}


\subsubsection{Синхронизация с КИС}
\label{bp:storege_integration}

\begin{enumerate}
\item Выгрузка справочных данных из систем \erp в систему \gofro производится автоматически.

\begin{enumerate}
\item
Справочник  \myobject{Номенклатура} в части материалов из системы \erp загрузится в Справочник  \myobject{Номенклатура} системы \gofro.
\end{enumerate}

\item 	

% \item
 Остатки материалов (бумага, картон) по всем складам на момент обмена из системы \erp загрузится в документ  \myobject{ВводОстатков} системы \gofro.
\end{enumerate}

% \item 	Выгрузка документов из системы \gofro в системы \erp и \stock производится автоматически по регламенту обмена.

 % \begin{enumerate}
 % \item Документ \myobject{ПоступлениеТМЦ} из системы \gofro  в документ \myobject{Поступление} системы  \erp.
 % \item Документ \myobject{СписаниеТМЦ} из системы \gofro выгружается в документ \myobject{СписаниеТМЦ} системы  \erp.
 % \item Документ \myobject{ПеремещениеТМЦ} из системы \gofro выгружается в документ \myobject{ПеремещениеТоваров} системы  \erp.
 % \item Документ \myobject{ИнвентаризацияТМЦ} из системы \gofro выгружается в документ \myobject{ИнвентаризацияТоваров} системы  \erp.

% Потребление материалов (бумага, картон) по гофроагрегату за смену из системы \gofro загрузится в документ \blue{\#Перемещение} системы \stock.

