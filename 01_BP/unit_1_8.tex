\newpage
\subsection{Описание бизнес-процесса «Выпуск готовой продукции и полуфабрикатов»}
\label{bp:production}

Модель бизнес-процессов представлена на Рис. \ref{pic:4_Output} Модель бизнес-процесса «Выпуск готовой продукции и полуфабрикатов».

\textbf{Цель}

Основной целью процесса «Производство продукции» является организация производства продукции.

\subsubsection{Сценарий ''Поступление сменного задания''}
\label{bp:production_1}

\begin{enumerate}

\item 	\planner ежедневно  передает планы производства для \master. Задания распечатываются при необходимости из системы \gofro формы \myform{Ф\_ЗаданиеНаГА}, \myform{Ф\_ЗаданиеНаЛинии}.
\item 	\planner передает задания на линии в системе \gofro в форме \myobject{НепрерывныйПлан}.
\item В системе \gofro \operator и \gaoperator самостоятельно подгружают задания, сформированные \planner.
\end{enumerate}


\subsubsection{Сценарий ''Распределение заданий внутри смены''}
\label{bp:production_2}

\begin{enumerate}

\item 	\planner  в системе \gofro при выполнении бизнес-процесса \textbf{«Планирование выпуска готовой продукции»} распределяет задания по гофроагрегатам и линиям переработки.
\end{enumerate}



\subsubsection{Сценарий ''Доведение заданий до операторов оборудования''}
\label{bp:production_3}

\begin{enumerate}
\item \gaoperator и \operator получают задания на работу в системе \gofro. %(при выполнении процедуры подгрузки списка задаемй из \#НепрерывныйПлан в документ выработки.
\item 	В каждом задании, получаемом в электронном виде в системе \gofro,  указаны все необходимые для настройки оборудования данные.
\end{enumerate}


\subsubsection{Сценарий ''Обеспечение сырьем''}
\label{bp:production_4}

% \todo{Будет ли у них учет сырья на мокрой части? Им точно нужно???}

\begin{enumerate}


% \item	Согласно процедуре «Закупки» \planner осуществляет закупку необходимого сырья: бумаги, картона, химикатов и дополнительных материалов: стрейч-пленка, ленты и др.
% \item	Отдел планирования  и \planner рассчитывает потребность в сырье и материалах в системе \gofro по бумаге и картону согласно процедуре «Планирование выпуска готовой продукции».
% \item	\planner передает в системе \gofro из формы \myform{НепрерывныйПлан} формирует план потребности в сырье \textit{Ф\_ОтчетПоНеобходимомуСырью} и передает \master.
% \item	\master передает план потребности в сырье \kladovshik.
% \item	\kladovshik склада сырья выдает бумагу и картон в производство по накладной и на основании потребности в сырье \textit{Ф\_ОтчетПоНеобходимомуСырью} согласно процедуры «Перемещение рулонов»
% \item	В зимнее время рулоны, поступающие на гофроагрегат, должны пройти акклиматизацию в цехе в течение 3 дней.



\item 	\planner при помощи системы \gofro рассчитывает потребность в сырье и материалах по бумаге и картону согласно процедуре \textbf{«Планирование выпуска готовой продукции»}.
\item 	\planner в системе \gofro в форме \myobject{НепрерывныйПлан} 
формирует план потребности в сырье \myform{Ф\_ОтчетПоНеобходимомуСырью} и передает пользователю \master.
\item	\master передает план потребности в сырье \driver и \linkoperator на мокрую часть гофроагрегата. 
\item	\driver перемещает с помощью погрузчика сырье в буфер гофроагрегата, откуда \linkoperator ставит сырье на раскат.
\item	\linkoperator при установке сырья на раскат гофроагрегата сканирует штрих-код на рулоне с помощью сканера штрих-кода в системе \syncro.
Система \syncro выгружает данные в систему \gofro в документе \myobject{СырьеДляВыработки} информацию по потребленному сырью. Система \gofro фиксирует факт постановки рулона.
% \item	\linkoperator после окончания работы с рулоном при сматывании его до конца в системе \gofro в документе \myobject{СырьеДляВыработки} указывают остаток = 0. 
% \item	\linkoperator при снятии остатка рулона с раската регистрирует остаток согласно процедуре \textbf{«Списание рулонов бумаги и картона»}.

% \item \linkoperator в течение смены на ”мокрой части” 
% оформляет бланк ”Учета срывов на мокрой части” вручную. 

\end{enumerate}




\subsubsection{Сценарий ''Обеспечение дополнительными материалами и инструментами''}
\label{bp:production_5}

\begin{enumerate}
\item	Согласно технологического режима производства необходимо наличие штанц-форм и/или клише (оснастка), краски. Подготовка к производству штанц-форм, клише и краски производится согласно процедуре \textbf{«Подготовка штанц-форм, клише, краски»}.

\item Краска поступает на предприятие в готовом виде, подачу краски на линии переработки осуществляется сотрудниками бригады. Остатки краски \master контролирует согласно утвержденному регламенту.

%\item Краска готовится технологами-колористами согласно планам и приносится к линиям переработки. Остатки краски технолог-колорист забирает с линий.
% и фиксирует
\end{enumerate}



\subsubsection{Сценарий ''Подготовка пакета документов для осуществления производства''}
\label{bp:production_6}

\begin{enumerate}
\item	В пакет документов для осуществления производства входят: заказ на производство, этикетка на паллету, технологическая карта.
\item \gaoperator получает пакет документов в электронном виде в системе \gofro в документе \myobject{ВыработкаГофроагрегата}: раскрои (заказы), технологическая карта, этикетка на каждый транспортный пакет (паллет).
\item \operator получает пакет документов в электронном виде в системе \gofro в документе \myobject{ВыработкаПоПереработке}: задания (заказы), технологическая карта, этикетка на каждый транспортный пакет.
\item \operator печатает бирку на готовую паллету в системе \gofro в документе \myobject{ВыработкаПоПереработке}. Бирка печатается со штрих-кодом, где указан номер заказа и количество продукции на паллете.

\end{enumerate}


\subsubsection{Сценарий ''Подготовка персонала''}
\label{bp:production_7}

\begin{enumerate}
\item Подготовка персонала осуществляется на основании рабочих инструкций.
\item 	С целью обеспечения качества производства, соблюдения технологических параметров разрабатываются инструкции по эксплуатации оборудования. Директор по производству несет ответственность за ее актуальность.


\end{enumerate}

\subsubsection{Сценарий ''Обслуживание производственного оборудования''}
\label{bp:production_8}

\begin{enumerate}
\item 	Для обеспечения безварийной работы технологического оборудования планируется техническое обслуживание, предупредительные и капитальные ремонты согласно утвержденного годового графика ППР.
\item 	В случае выхода из строя оборудования операторы оповещают дежурный технический персонал. В экстренном порядке производится ремонт оборудования. 
\item 	\gaoperator фиксирует простой по причинам неполадок оборудования в системе \gofro в документе \myobject{ВыработкаГофроагрегата}.
\item 	\operator фиксирует простой по причинам неполадок оборудования в системе \gofro в документе \myobject{ВыработкаПоПереработке}.
\item 	В конце смены \master проверяет простои оборудования в системе \gofro в отчете \myform{Ф\_ПростоиОборудования}.


\end{enumerate}





\subsubsection{Сценарий ''Выпуск заготовок на гофроагрегате''}
\label{bp:production_9}

\begin{enumerate}
\item  \gaoperator в системе \gofro создает новый документ \myobject{ВыработкаГофроагрегата}. Если документ был создан заранее \planner, то \gaoperator открывает его.
\item  \gaoperator указывает бригаду и смену в системе \gofro.
\item  \gaoperator в системе \gofro нажимает кнопку \myform{ЗаполнитьПоНП}. Система \gofro отображает пользователю из непрерывного плана список заданий на указанный гофроагрегат, запланированные пользователем \planner. \gaoperator в списке требуемых к выполнению заданий должен отметить те, которые следует загрузить в документ, не меняя очередность следования раскроев. 
% \item  \gaoperator в системе \gofro нажимает кнопку \myform{ЗаполнитьПоПлану} Система \gofro отображает пользователю из плана список заданий на указанный гофроагрегат, запланированные пользователем \planner. 
\item Система \gofro заполняет табличную часть в документе \myobject{ВыработкаГофроагрегата} списком запланированных к выпуску раскроев, оставляя пустыми колонки с фактическими объемами. Таким образом \gaoperatorполучает в системе \gofro в документе \myobject{ВыработкаГофроагрегата} список запланированных к выпуску раскроев.
\item  \gaoperator	для каждого заказа  в системе \gofro печатает внутренние бирки (для полуфабрикатов) по команде \myform{БиркаВнутрицеховая}. Бирки можно распечатать как для всех позиций выработки, так и для выделенных строк.
\item  \gaoperator в системе \gofro по команде \myform{Бирка} печатает бирки на готовую продукцию (товарный гофрокартон).%\gaoperator в системе \gofro по команде \$Талон печатает сопроводительные талоны на готовую продукцию (товарный гофрокартон). 
% Система \gofro формирует список номеров паллет. 
Система \gofro по команде \myform{Печать}
\ifnum\IsScancode=1
   выводит на печать форму бирки с штрих-кодом, 
   содержащим % уникальный 
   номер паллеты.
\else
   выводит на печать форму бирки на готовую продукцию. 
\fi
\item  \gaoperator	при необходимости в системе \gofro  нажимает кнопку \myform{ТКПолная} для вывода на экран технологической  карты по заказу из выделенного раскроя. Система \gofro откроет технологическую карту в печатном виде.
\item  \gaoperator выделяет раскрои для выгрузки в систему смены заказа \syncro  гофроагрегата Fosber и выгружает из системы \gofro  по команде \myform{ВыгрузитьРаскрои}.
\item  Система \gofro подключается к системе \syncro.
\item 	Система \gofro опрашивает систему смены заказа \syncro, получает информацию по работе гофроагрегата и автоматически загружает информацию о фактически выработанных объемах по раскрою, браке в заготовках, браке при перестроении раскроя. Все данные обязательно загружаются с учетом номера производственного заказа.
\item	Информация по простоям будет Формируется в таблице «Журнал работы оборудования» с указанием времени начала и окончания останова и причины останова. \gaoperator в системе \gofro уточняет причину и время останова. 
Таблица заполняется автоматически по данным работы системы \syncro.
% Таблица «Журнал работы оборудования» заполняется \gaoperator путем нажатия кнопок <<Контроль переналадки>>, <<Контроль паузы>>, <<Контроль простоя>>.
\item	\gaoperator в случае возникновения брака  в системе \gofro в документе \myobject{ВыработкаГофроагрегата} в таблице «Брак» указывает количество брака (в штуках) по заказу и причину возникновения брака из справочника.
\item Бракованные заготовки вывозятся водителем погрузчика к шредеру.



\item После смены заказа (раскроя) система \gofro автоматически проводит документ \myobject{ВыработкаГофроагрегата}. При проведении документа данные о выработке будут доступны пользователям в системе \gofro . 
\item 	 По окончании смены или в течение смены \gaoperator заполняет в системе \gofro  в журнале работы оборудования реальные причины простоя для позиций, где был указан признак «Причина не определена». В системе \gofro  в журнале работы выдается подсказка о наличии таких простоев.
\item 	\gaoperator в течение смены в системе \gofro  указывает в документе \myobject{ВыработкаГофроагрегата} список работников, должность и время их работы.
\item	Картон и бумага учитываются в производство на гофроагрегате в соответствии с процедурой \ref{bp:production_10} «Списание рулонов бумаги и картона».
\item	\master в конце смены в системе \gofro  открывает форму \myobject{ВыработкаГофроагрегата}
\master проверяет отчеты производства по каждой линии, устанавливает признак «Проверено».
\item		 После этого вносить изменения в документ сможет только пользователь, обладающий соответствующими правами. Установка данного признака позволяет обезопасить документ в системе \gofro от последующих несанкционированных изменений пользователями.
\item	После проведения документ \myobject{ВыработкаГофроагрегата} будет автоматически выгружен в систему \erp.
\item Ответственным за верное заполнение отчетов по выработке является \master.





\end{enumerate}



\subsubsection{Сценарий ''Списание рулонов в производство''}
\label{bp:production_10}

\begin{enumerate}
\item Первичное списание рулонов бумаги и картона производится  в системе \gofro на гофроагрегате согласно процедуры \ref{bp:storage_8} \textbf{«Списание рулонов бумаги и картона»}.
\end{enumerate}



\begin{comment} % Нет подключения к оборудованию


\subsubsection{Сценарий ''Производство продукции на линиях переработки c подключаемым оборудованием''}
\label{bp:production_20}

\begin{enumerate}
\item \operator на линии переработки заносит в системе \gofro выработку по каждому из станков и каждому переделу.
\item \operator создает в системе \gofro новый документ \myobject{ВыработкаПоПереработке}, указывает  бригаду и смену. Если документ был создан заранее \planner, то открывает его.
\item \operator нажимает кнопку \myform{ЗаполнитьПоПлану} в документе \myobject{ВыработкаПоПереработке}. Система \gofro отображает пользователю из плана список заданий на указанный станок, которые запланировал \planner. 
\item Система \gofro заполняет документ списком запланированных к выпуску заказов, оставляя пустыми колонки с фактическими объемами.
\item \operator	При необходимости  в системе \gofro  нажимает кнопку \myform{ТК} для вывода на экран технологической карты по выпускаемому заказу (-ам). Система \gofro откроет технологическую карту в печатном виде.
\item При необходимости \operator в системе \gofro печатает внутренние бирки (для полуфабрикатов) по команде \myform{БиркаВнутрицеховая}. Бирки можно распечатать как для всех позиций выработки, так и для выделенных строк. Данные о контрагенте и характеристики изделия заполняются автоматически.
\item \operator  в системе \gofro по команде \myform{Бирка} печатает бирки на готовую продукцию. %\operator в системе \gofro по команде \myform{Талон} печатает сопроводительные талоны на готовую продукцию (товарный гофрокартон). 
Система \gofro формирует список номеров паллет. Система \gofro по команде \myform{Печать}
% Есть учет ГП по штрих-кодам
\ifnum\IsScancode=1
   выводит на печать форму бирки с штрих-кодом, 
   содержащим уникальный номер паллеты.
\else
   выводит на печать форму бирки на готовую продукцию. 
\fi


\item \operator в начале выработки заказа  запускает в системе \gofro заказ в наладку. Система \gofro записывает время начала наладки. 
%\item \operator после окончания наладки  запускает в системе \gofro заказ в работу. Система \gofro записывает время начала работы, начальное состояние датчика заготовок из OPC-библиотеки, куда информация загружается с контролера линии, определяет количество использованных заготовок на наладку. Данные о фактически выработанных изделиях загружаются в систему \gofro автоматически из OPC-библиотеки.
\item \operator может исправить данные по фактически выработанным заготовкам, при этом достоверными будут данные, указанные пользователем в системе \gofro.
\item	В случае возникновения простоя %(не поступают данные с контролера линии в течение периода времени, указанного в оборудовании) 
в системе \gofro в документе \myobject{ВыработкаПоПереработке} на вкладке «Журнал работы оборудования» автоматически указывается текущее время начала и окончания останова. Затем  \operator указывает причину останова из классификатора. Таблица «Журнал работы оборудования» заполняется \gaoperator путем нажатия кнопок <<Контроль переналадки>>, <<Контроль паузы>>, <<Контроль простоя>>.

%\item \operator в течение смены по каждому заказу печатает из системы \gofro бланк чек лист контроля качества (форма Ф\_Чеклистконтролякачествао).

\item \operator в случае возникновения брака  в системе \gofro в документе \myobject{ВыработкаПоПереработке} в таблице «Брак» указывает количество (в штуках) отбракованных изделий по заказу и причину брака выбором из справочника.
\item Бракованные заготовки вывозятся водителем погрузчика к шредеру.
\item По окончании выработки по заказу \operator регистрирует в системе \gofro факт окончания выработки по заказу.
\item	После смены заказа система \gofro проводит документ \myobject{ВыработкаПоПереработке}. При проведении документа данные о выработке будут записаны в регистр  в системе \gofro. После проведения данные выработки становятся доступны для анализа всем пользователям системы \gofro в режиме онлайн.
\item \planner при последующем планировании заказов   будет учитывать выработанные объемы, и система \gofro в план поставит только оставшийся объем.
\item \operator по окончании смены или в течение смены  заполняет в системе \gofro в журнале работы оборудования в документе \myobject{ВыработкаПоПереработке} реальные причины простоя для позиций, где был указан признак «Причина не определена». В системе \gofro в журнале работы выдается подсказка о наличии таких простоев.
\item \master  в конце смены под своим логином в системе \gofro открывает форму документа \myobject{ВыработкаПоПереработке}, проверяет отчеты производства по каждой линии, устанавливает в документах признак «Проверено». 

% \item Система \erp возвращает сообщения о возможности закрытия смены или ошибки.
% \item \master исправляет ошибки и закрывает смены в системе \gofro. 
%\item Документ \myobject{ВыработкаПоПереработке} с признаком «Проверено» будет выгружен автоматически в систему \erp.
\item После этого вносить изменения в документ сможет только пользователь, обладающий соответствующими правами. Установка данного признака позволяет обезопасить документ в системе \gofro от последующих несанкционированных изменений пользователями.

\item Ответственным за верное заполнение отчётов по выработке является \master.


\end{enumerate}



\end{comment}





\subsubsection{Сценарий ''Производство продукции на отдельных станках переработки и ручных операциях без подключаемого оборудования''}
\label{bp:production_21}

\begin{enumerate}
\item	\operator заносит в системе \gofro выработку по каждому из станков и каждому переделу.
\item	\operator открывает смену в системе \gofro, создает в системе \gofro новый документ \myobject{ВыработкаПоПереработке}, указывает  бригаду и смену. Если документ был создан заранее \planner, то открывает его.
\item	\operator нажимает кнопку \myform{ЗаполнитьПоНП} в документе \myobject{ВыработкаПоПереработке}. Система \gofro отображает пользователю список заданий для указанного станка, которые запланировал \planner. 
\item	\operator в списке требуемых к выполнению заданий отмечает те заказы, которые следует загрузить в документ. Система \gofro заполняет документ списком запланированных к выпуску заказов, оставляя пустыми колонки с фактическими объемами.
\item	При необходимости \operator в системе \gofro  нажимает кнопку \myform{ТК} для вывода на экран технологической карты по выпускаемому заказу (-ам). Система \gofro откроет печатную форму технологической карты.
\item При необходимости \operator в системе \gofro печатает внутренние бирки (для полуфабрикатов) по команде \myform{БиркаВнутрицеховая}. Бирки можно распечатать как для всех позиций выработки, так и для выделенных строк. Данные о контрагенте и характеристики изделия заполняются автоматически.
\item	\operator в системе \gofro по команде \myform{Бирка} печатает бирки на готовую продукцию. 
% Система \gofro формирует список номеров паллет. 
Система \gofro по команде \myform{Печать} 
\ifnum\IsScancode=1
   выводит на печать форму бирки с штрих-кодом, 
   содержащим
   % уникальный 
   номер паллеты.
\else
   выводит на печать форму бирки на готовую продукцию. 
\fi
\item	В течение смены \operator в системе \gofro в документе \myobject{ВыработкаПоПереработке}  заносит информацию о фактически выработанных объемах, браке. Все данные обязательно вносятся с учетом номера производственного заказа.
\item	В случае возникновения простоя \operator  в системе \gofro в документе \myobject{ВыработкаПоПереработке} на вкладке «Журнал работы оборудования» указывает время начала и окончания останова и причину останова по команде \myform{УказатьПростой}.
\item	В случае возникновения брака \operator  в системе \gofro в документе \myobject{ВыработкаПоПереработке}  в таблице «Брак» указывает количество отбракованных изделий по заказу и причину брака выбором из справочника
\item	\operator  проводит документ \myobject{ВыработкаПоПереработке}. При проведении документа данные о выработке будут записаны в регистр  в системе \gofro. После проведения данные выработки становятся доступны для анализа всем пользователям системы \gofro в режиме онлайн.
\item	При последующем планировании заказов выработанные объемы \planner будет учитывать, и система \gofro допланирует только оставшийся объем.
\item	По окончании смены \operator  заполняет в системе \gofro в журнале работы оборудования в документе \myobject{ВыработкаПоПереработке}  реальные причины простоя для позиций, где был указан признак «Причина не определена».
\item	\master  в конце смены в системе \gofro открывает форму документа \myobject{ВыработкаПоПереработке}.  \master  проверяет отчеты производства по каждой линии, устанавливает признак «Проверено». 
\item	Документ \myobject{ВыработкаПоПереработке} с признаком «Проверено» по регламенту будет выгружен в систему \erp.
\item	 После этого вносить изменения в документ сможет только пользователь, обладающий соответствующими правами. Установка данного признака позволяет обезопасить документ в системе \gofro от последующих несанкционированных изменений пользователями.
\item	Ответственным за верное заполнение отчётов по выработке является \master.



\end{enumerate}





\subsubsection{Сценарий ''Управление несоответствующей продукцией''}
\label{bp:production_30}

\begin{enumerate}
\item	Несоответствующая продукция, образовавшаяся в процессе производства заготовок или готовой продукции, идентифицируется биркой с пометкой «Брак» и определяется в зону брака до принятия решения:

\begin{enumerate}
    \item исправить данный брак;
    \item продать продукцию по меньшей цене по другому заказу;
    \item списать на заготовки и упаковку;
    \item списать на макулатуру.
\end{enumerate}
\item	\gaoperator учитывает брак в системе \gofro в документе \myobject{ВыработкаГофроагрегата}.
\item	\operator на  рабочих местах учитывает брак в системе \gofro в документе \myobject{ВыработкаПоПереработке}.
% \item	На прочих линиях брак учитывает \master  в системе \gofro в документе \myobject{ВыработкаПоПереработке}.
\item	\master  получает в конце смены в системе \gofro отчет по браку по форме \myform{Ф\_БракПоСменам}.



\end{enumerate}



\subsubsection{Сценарий ''Упаковка и идентификация''}
\label{bp:production_40}

\begin{enumerate}
%\item РОМА. Дописать печать сопроводительных талонов???

\item	Требования по упаковке указываются в технологической карте изделия и доступны в системе \gofro в справочнике \myobject{ТехнологическаяКарта}.
\item	После изготовления заготовок на гофроагрегате сформированные паллеты перемещаются в отведенное место для временного хранения. На все паллеты \gaoperator вывешивает этикетку по форме \myform{Ф\_БиркаЦеховая} с указанием номера заказа, технологической карты, количества на паллете, даты выработки.
\item	Готовая продукция формируется в паллеты, если другое не указано в технологической карте изделия. При упаковке на паллеты \gaoperator или \operator  вывешивает этикетку по форме \myform{Ф\_Бирка} с указанием номера заказа, технологической карты, количества на паллете, даты выработки. 
% Дополнительно для каждой паллеты \gaoperator или \operator печатает сопроводительный талон по форме \textit{Ф\_Талон} с указанием номера заказа, артикула,количества на паллете, даты выработки.
\item	Готовая продукция в паллетах упаковывается на линиях упаковки. \planner НЕ ПЛАНИРУЕТ работу линий упаковки. Линия упаковывает все паллеты, поступающие со всех линий переработки по разным заказам. 
\end{enumerate}








\subsubsection{Синхронизация с КИС}
\label{bp:production_exchange}

\begin{enumerate}
\item	Выгрузка данных из \gofro в \erp производится автоматически.

\begin{enumerate}

\item	Документ \myobject{ВыработкаПоПереработке}, \myobject{ВыработкаГофроагрегата} из системы \gofro выгружаются в систему \erp в документ \myobject{ОтчетПроизводстваЗаСмену} автоматически по регламенту.

\end{enumerate}

\end{enumerate}