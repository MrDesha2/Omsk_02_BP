\subsection{Описание бизнес-процесса «Отгрузка готовой продукции»}
\label{bp:goods}

\subsubsection{Сценарий ''Прием готовой продукции на склад}
% . Вариант с Штрих-кодированием''}
\label{bp:goods_1}


% % **************************************************************************************************
% \ifnum\IsScancode=1    % Нужно штрих-кодирование ГП

\begin{enumerate}

\item  \gaoperator печатает бирки из системы \gofro рабочее место \myobject{ВыработкаГофроагрегата} из документа \myobject{ФормированиеПаллет}  на товарный картон и прикрепляет бирку с уникальным штрих-кодом 
номенклатуры и характеристики
на каждый паллет готовой продукции.
\item  \operator печатает бирки из системы \gofro рабочее место \myobject{ВыработкаПоПереработке} из документа \myobject{ФормированиеПаллет}  на готовую продукцию и прикрепляет бирку %с уникальным штрих-кодом 
на каждый паллет готовой продукции с штрих-кодом номенклатуры.
%  \item  \gaoperator печатает сопроводительные талоны из системы \gofro рабочее место \myobject{ВыработкаПоПереработке}  прикрепляет  на каждый паллет готовой продукции. \todo{Это откуда-то осталось. У них есть такие талоны???}
% \item  \operator печатает сопроводительные талоны из системы \gofro рабочее место \myobject{ВыработкаПоПереработке}  прикрепляет  на каждый паллет готовой продукции. \todo{Это откуда-то осталось. У них есть такие талоны???}

\item \driver перемещает готовую продукцию на склад готовой продукции на упакованных паллетах.
\item \driver в течение дня выполняет приемку готовой продукции сканированием паллет через ТСД в системе \erp. При этом в системе \erp создается документ \myobject{Перемещение ГП}
\driver сканирует каждый паллет сканером ТСД.

\item  \kladovshik в течение смены осуществляет прием готовой продукции по факту в системе \erp в отчете по готовой продукции. 
\item  \kladovshik  в конце смены  передает сдаточные акты в бухгалтерию
%\item  \kladovshik считывает штрих-код паллеты сканером штрих-кода в системе \erp. 

%\item Штрих-коды паллет по готовой продукции загружаются автоматически из системы \gofro в систему \erp.

%\item Каждый загруженный из \gofro штрихкод является серией номенклатуры в системе \erp.

% \item	\kladovshik получает от !Водителя приходные документы от поставщика.

% \item Приемка готовой продукции в системе \erp выполняется автоматически при загрузке в конце смены документов \myobject{ВыработкаГофроагрегата} и \myobject{ВыработкаПоПереработки} из системы \gofro в систему \erp (см. п. \ref{bp:production_exchange}).

\end{enumerate}


\subsubsection{Сценарий ''Планирование отгрузки''}
\label{bp:goods_2}

\begin{enumerate}
    \item Возврат готовой продукции от покупателя выполняет в системе \auditor в системе \erp.
    \item \auditor дублирует вручную документы по возврату в системе \buh.
    
\end{enumerate}



\subsubsection{Сценарий ''Планирование отгрузки''}
\label{bp:goods_2}


\begin{enumerate}
    \item \manager формирует в системе \erp документ \myobject{ЗаявкаНаОтгрузку}, заполняя его готовыми к отгрузке позициями и адресом доставки. \manager создает документ вручную.
    \item \manager в форме отчета \myform{ПортфельЗаказов} в системе \gofro определяет состояние заказа, объем  выработки, количество продукции.
    \item	\manager определяет в системе \erp остатки готовой и отгруженной продукции.
    %, сданной на склад и количество отгруженной продукции.
    \item	\manager для планирования отгрузки согласует с заказчиком дату отгрузки готовой продукции в системе \erp документ \myobject{ЗаявкаНаОтгрузку}.
    \item	При создании документа в системе \erp \manager вручную определяет объемы для отгрузки по остаткам готовой продукции.
    % \item	\manager меняет статус документа на “ДляЛогиста”, проводит документ #ЗаявкаНаОтгрузку.
\end{enumerate}



% \subsubsection{Сценарий ''Выдача задания на отгрузку''}
% \label{bp:goods_3}


% \begin{enumerate}
% % \item	!МенеджерПоЛогистике получает заявки на отгрузки в системе @ГТ со статусом “ДляЛогиста”.
% \item	\manager 
% % при получении заявок от всех менеджеров 
% начинает планировать транспорт в системе \gofro в форме \myobject{График отгрузки} и формирует в системе \gofro отчет \myobject{ГрафикОтгрузки}.
% \item	\manager в графике отгрузки указывает время подачи машины, водителя и машину.
% \item	После согласования условий отгрузки в документе \myobject{ЗаявкаНаОтгрузку} \manager меняет статус на “Одобрен в отгрузку”, печатает при необходимости и передает на склад готовой продукции.
% \item	\kladovshik в системе \gofro видит заявки на отгрузку в журнале документов  \myobject{ЗаявкаНаОтгрузку} или получает от \manager печатный отчет \myform{ГрафикОтгрузки}, список заявок на отгрузку.

% \end{enumerate}


\subsubsection{Сценарий ''Отгрузка ГП со склада''}
\label{bp:goods_4}

Процесс не меняется и выполняется в \erp.

% \begin{enumerate}
% \item	Отгрузка готовой продукции в системе \gofro  осуществляется \kladovshik согласно печатного документа \myobject{ЗаявкаНаОтгрузку} со статусом «Одобрен в отгрузку».
% \item	\kladovshik отгружает готовую продукцию согласно заявки на отгрузку, отмечает погруженные позиции и количество.
% \item	\kladovshik в документе \myobject{ЗаявкаНаОтгрузку} вручную заполняет колонки с фактически погруженным количеством и на основании заявки на отгрузку  создает в системе \erp документ \myobject{РеализацияТМЦ}.
% \item	Печатный документ \kladovshik передает в бухгалтерию.
% \item   \auditor  создает в системе \erp документ \myobject{Реализация} и выписывает сопроводительные документы.
% % \item   !МенеджерПоСкладскойЛогистике или !Кладовщик (в случае отсутствия !МенеджерПоСкладскойЛогистике) печатает сопроводительные документы в системе @1С:Бухгалтерия.
% \end{enumerate}



\subsubsection{Сценарий ''Завершение задания на отгрузку''}
\label{bp:goods_5}

Процесс не меняется и выполняется в \erp.


% \begin{enumerate}

% \item После выполнения \myobject{ЗаявкаНаОтгрузку} \kladovshik сообщает \manager об окончании погрузки.
% \item \manager в системе \gofro изменяет статус в документе \myobject{ЗаявкаНаОтгрузку} на «Выполнен».
% \end{enumerate}





\subsubsection{Сценарий ''Возврат продукции''}
\label{bp:goods_6}

\begin{enumerate}

\item Первичный учет возврата готовой продукции ведется в системе \erp. 
\item Процесс не меняется.
% \item \kladovshik контролирует количество и номенклатуру ТМЦ, после чего сравнивает с возвратными документами
% \item	\kladovshik при возврате в системе \gofro создает документ \myobject{ПоступлениеТМЦ} (статус “Возврат ГП”), при этом указывает поставщика и склад-получатель.
% \item	\kladovshik по каждой продукции системе \gofro указывает позицию из справочника \myobject{Номенклатура}, количество и цену.
% \item	\kladovshik проводит документ \myobject{ПоступлениеТМЦ}.

\end{enumerate}


\subsubsection{Сценарий ''Синхронизация с КИС''}
\label{bp:goods_exchange}

%\begin{enumerate}
%\item Не производится.
%\end{enumerate}

\begin{enumerate}
\item Выгрузка данных о выпуске ГП не производится.
% в систему \erp из системы \gofro производится автоматически по регламенту обмена согласно п. \ref{bp:production_exchange}.

%\begin{enumerate}
%\item
%Данные автоматически выгрузятся из системы \gofro в документ %\myobject{ОтчетПроизводстваЗаСмену} системы \erp.  
%\end{enumerate}

% \item 	Выгрузка документов из системы \gofro в системы \erp и \stock производится автоматически по регламенту обмена.

% \begin{enumerate}
% \item
% Потребление материалов (бумага, картон) по гофроагрегату за смену из системы \gofro загрузится в документ \blue{\#Перемещение} системы \stock.
% \end{enumerate}

\end{enumerate}

% \textbf{ТОЧНО У НАС?}

% \begin{enumerate}
% \item Остатки готовой продукции по складу менеджеры и специалисты ОМиП ежедневно просматривают в \erp.


% \item \manager  формирует в системе \erp документ #ЗаявкаНаОтгрузку, заполняя его готовыми к отгрузке позициями и адресом доставки. !МенеджерПоПродажам создает документ вручную, либо на основании данных отчета $ПортфельЗаказов, либо на основании документа #Заявка.
% 2.	В форме отчета $ПортфельЗаказов в системе @ГТ определяет состояние заказа, объем выработки, количество продукции, сданной на склад и количество отгруженной продукции.
% 3.	!МенеджерПоПродажам для планирования отгрузки согласует с заказчиком дату отгрузки готовой продукции.
% 4.	При создании документа в системе @ГТ !МенеджерПоПродажам вручную определяет объемы для отгрузки по остаткам готовой продукции.
% 5.	!МенеджерПоПродажам меняет статус документа на “ДляЛогиста”, проводит документ #ЗаявкаНаОтгрузку.

% \end{enumerate}