\section{Введение}

\subsection{Общие сведения}

В настоящем документе представлены результаты модернизации бизнес-процессов предприятия \FIRMA, связанные с внедрением ПС ПП «Гофротара», полученные на основе анализа, выполненного в рамках Договора \agreement. 


\newpage
\subsection{Роли}

\textbf{Роль (Бизнес-роль)} – совокупность задач, обязанностей и полномочий, связанных с конкретным участником бизнес-процесса, который несет ответственность за определенный набор функций, связанных между собой, и наделен определенным уровнем полномочий, позволяющим реализовать поставленные задачи.

Бизнес-роль присваивают сотруднику в зависимости или вне зависимости от занимаемой им должности.  

Бизнес-роли в бизнес-процессах разделяют на 2 типа: верхнего уровня и нижнего уровня.

\begin{itemize}
\item{\textbf{Роли нижнего уровня} – решают задачи, неся ответственность только за  выполнение своих функций.}\
\item{\textbf{Роли верхнего уровня} – решают задачи, неся ответственность за выполнение функций, назначенных ролям нижнего уровня.} \
\end{itemize}

При внедрении в компании процессно-ориентированного управления создают ролевую концепцию управления бизнес-процессами. Для этого в локальном нормативном акте описывают бизнес-роли и формулируют к ним основные требования, указывая сферу полномочий, ответственности, взаимосвязь с должностями, которые присутствуют в штатном расписании. 

Бизнес-роли должны иметь отношение и к самому процессу и к управлению процессами. 

Согласно ролевой концепции управления бизнес-процессами, участники исполняют определенные функции в рамках конкретного бизнес-процесса.

Бизнес-роли, предлагаемые в рамках модернизации бизнес-процессов предприятия, представлены в таблице \ref{bp:roles}.


\newpage

\begin{longtable}{|p{69mm}|p{100mm}|}
\hline
{\bf \parbox[c][5mm]{69mm}{\centering Наименование}} & {\bf \parbox[c]{100mm}{\centering Описание}} \\
\hline
{{\bf \parbox[c][15mm]{69mm}{\operator }}} & 
{\it Группа пользователей: ПодсистемаВыработка. Машинист линии переработки } \\
\hline
{\it {\bf \parbox[c][10mm]{69mm}{\gaoperator }}} & {\it Группа пользователей: ПодсистемаВыработка. Машинист гофроагрегата} \\
\hline
{\it {\bf \parbox[c][10mm]{69mm}{\kladovshik }}} & {\it Группа пользователей: ПодсистемаСклады. Кладовщик склада сырья и готовой продукции} \\
\hline
{\it {\bf \parbox[c][10mm]{69mm}{\manager }}} & {\it Группа пользователей: ПодсистемаПродажи, ПодсистемаТК. Менеджер отдела маркетинга и продаж, ассистент отдела маркетинга и продаж, ассистент менеджера по продажам} \\
\hline
{\it {\bf \parbox[c][10mm]{69mm}{\planner }}} & {\it Группа пользователей: ПодсистемаПланирование. Специалист по планированию производства } \\
\hline
{\it {\bf \parbox[c][15mm]{69mm}{\tehnolog }}} & {\it  Группа пользователей: ПодсистемаТК. Технолог, дизайнер, конструктор} \\
 \hline
{\it {\bf \parbox[c][15mm]{69mm}{\preproductionspecialist }}} & {\it  Группа пользователей: ПодсистемаТК. Специалист по подготовке производства} \\
%\hline
%{\it {\bf \parbox[c][10mm]{69mm}{\montaznik }}} & {\it  Группа пользователей: ПодсистемаТК. Слесарь-ремонтник монтажных форм, слесать-ремонтник печатной оснастки.} \\
% \hline
% {\it {\bf \parbox[c][10mm]{69mm}{\supplier}}} & {\it Группа пользователей: ПодсистемаСклады.
% Специалист по закупкам.}
% \\
% \hline
% {\it {\bf \parbox[c][10mm]{69mm}{\logistician}}} & {\it Группа пользователей: ПодсистемаПродажи, ПодсистемаСклады, ПодсистемаТранспорт
% Инженер отдела логистики, инженер по транспорту} \\
% & {Начальник отдела логистики} \\
\hline
{\it {\bf \parbox[c][15mm]{69mm}{\master }}} & {\it Группа пользователей: ПодсистемаВыработка. Мастер участка, который имеет право принимать выработку производственного оборудования и проверять сменные отчеты} \\
\hline
{\it {\bf \parbox[c][15mm]{69mm}{\processengineer}}} & {\it Группа пользователей: ПодсистемаВыработка. Директор по производству, роль верхнего уровня - контроль за исполнением производственных бизнес-процессов} \\
% \hline
% {\it {\bf \parbox[c][10mm]{69mm}{\prodtehnolog}}} & {\it Группа пользователей: ПодсистемаТК, ПодсистемаВыработка} \\
% Технолог производства, отвечающий за вопросы соблюдения технологии в процессе производства .}  \\
\hline
{\it {\bf \parbox[c][10mm]{69mm}{\auditor}}} & {\it Группа пользователей: Администратор. Бухгалтерия предприятия. Экономисты} \\
\hline
{\it {\bf \parbox[c][10mm]{69mm}{\director}}} & {\it  Группа пользователей: Администратор. Директор предприятия} \\
% \hline
% {\it {\bf \parbox[c][10mm]{69mm}{\laborant}}} & {\it  Группа пользователей: ПодсистемаКонтрольКачества. Лаборант отдела контроля качества.} \\
\hline
{\it {\bf \parbox[c][10mm]{69mm}{\laborant}}} & {\it  Группа пользователей: ПодсистемаКонтрольКачества. Контролер по качеству.} \\
\hline
\caption{Роли в процессах}\label{bp:roles}
\end{longtable}  
\normalsize



\subsection{ИТ-системы и сущности}

\begin{longtable}{|p{69mm}|p{100mm}|}
\hline
{\bf \parbox[c][5mm]{69mm}{\centering ИТ-системы, Сущности}} & {\bf \parbox[c]{100mm}{\centering Описание}} \\
\hline
\buh & {Информационная система  1С: Предприятие 8.3. Конфигурация «Бухгалтерия предприятия» }\\

\hline
\myobject{ПеремещениеТоваров} & Складской документ для регистрации факта перемещения товаров с одного склада на другой. \\
% \\
% \hline
% \myobject{ЗаказКлиента} & Документ ''Заказ клиента'' -- это запрос клиента на поставку ему товаров или оказание услуг в установленные сроки.  
% \\
% \myobject{РасходнаяНакладная} & Складской документ для регистрации факта отгрузки товаров со склада покупателю.
\hline
\myobject{ОтчетПроизводстваЗаСмену} & Документ предназначен для списания материалов с баланса складов и цеховых кладовых, а также работ с баланса подразделений, и их отнесения на выпущенную продукцию. Помимо материалов и работ в документе указываются выполненные работы сотрудников, которые требуется включить в себестоимость продукции. Документ является распоряжением на оформление документа выработки сотрудников.\\
% \hline
% \myobject{ПоступлениеТМЦ} & Документ ''Приобретение
% товаров и услуг'' предназначен для отражения различных операций по поступлению товаров и услуг. \\

%  \\
% \hline
% \myobject{ПередачаМатериаловВПроизводство} & Передача материалов в производство
% Документ регистрирует факты выдачи материалов со склада в производство.

\hline
\myobject{Контрагенты} & Справочник предназначен для хранения списка контрагентов. Контрагенты – это поставщики и покупатели, организации и частные лица.
\\
\hline
\myobject{Договоры} & Справочник предназначен для хранения договоров, заключенных с контрагентами.
\\
\hline
\myobject{Номенклатура} & Справочник предназначен для хранения информации о товарах, комплектах, наборах, готовой продукции, возвратной таре, полуфабрикатах, материалах, услугах, оборудовании.
\\
\hline
\myobject{Организация} & Справочник предназначен для хранения списка юридических лиц, входящих в состав предприятия (группы), а также хранения постоянных сведений о них. В этом же справочнике хранятся и сведения об индивидуальных предпринимателях, учет по которым ведется в программе.\\
\hline
\caption{Система \buh}\label{bp:system1}
\end{longtable}  
\normalsize




\begin{longtable}{|p{69mm}|p{100mm}|}
\hline
{\bf \parbox[c][5mm]{69mm}{\centering ИТ-системы, Сущности}} & {\bf \parbox[c]{100mm}{\centering Описание}} \\
\hline
\erp & {Информационная система 1С: Предприятие 8.3. Конфигурация «Управление производственными предприятием»}\\


\hline
\myobject{Заказ покупателя} & Документ «Заказ покупателя» (Заказ покупателя) предназначен для оформления предварительной договорённости с покупателем о намерении приобрести готовую продукцию под свои требования.
\\
\hline
\myobject{Заказ} & Документ «Заказ» (Заказ производства) предназначен для оформления требования для производства заданной продукции в указанном объеме к определенному сроку.
\\
\hline
\myobject{ПеремещениеТоваров} & Складской документ для регистрации факта перемещения товаров с одного склада на другой. \\
% \\
% \hline
% \myobject{ЗаказКлиента} & Документ ''Заказ клиента'' -- это запрос клиента на поставку ему товаров или оказание услуг в установленные сроки.  
% \\
% \myobject{РасходнаяНакладная} & Складской документ для регистрации факта отгрузки товаров со склада покупателю.
\hline
\myobject{ОтчетПроизводстваЗаСмену} & Документ предназначен для списания материалов с баланса складов и цеховых кладовых, а также работ с баланса подразделений, и их отнесения на выпущенную продукцию. Помимо материалов и работ в документе указываются выполненные работы сотрудников, которые требуется включить в себестоимость продукции. Документ является распоряжением на оформление документа выработки сотрудников.\\
\hline
\myobject{ПоступлениеТМЦ} & Документ ''Приобретение
товаров и услуг'' предназначен для отражения различных операций по поступлению товаров и услуг. \\
\hline
\myobject{РаспоряжениеНаОтгрузку} & Документ, предназначенный для выдачи распоряжения на отгрузку готовой продукции от менеджера, логисту и складским службам.
%  \\
% \hline
% \myobject{ПередачаМатериаловВПроизводство} & Передача материалов в производство
% Документ регистрирует факты выдачи материалов со склада в производство.
\\
\hline
\myobject{Контрагенты} & Справочник предназначен для хранения списка контрагентов. Контрагенты – это поставщики и покупатели, организации и частные лица.
\\
\hline
\myobject{Договоры} & Справочник предназначен для хранения договоров, заключенных с контрагентами.
\\
\hline
\myobject{Номенклатура} & Справочник предназначен для хранения информации о товарах, комплектах, наборах, готовой продукции, возвратной таре, полуфабрикатах, материалах, услугах, оборудовании.
\\
\hline
\myobject{Организация} & Справочник предназначен для хранения списка юридических лиц, входящих в состав предприятия (группы), а также хранения постоянных сведений о них. В этом же справочнике хранятся и сведения об индивидуальных предпринимателях, учет по которым ведется в программе.
\\
\hline
\myobject{Технологическая карта}  & Документ предназначен для хранения требований по изготовлению продукции. Содержит информацию по характеристике номенклатуры.
\\
\hline
\caption{Система \erp}\label{bp:system2}
\end{longtable}  
\normalsize





\begin{longtable}{|p{69mm}|p{100mm}|}
\hline
{\bf \parbox[c][5mm]{69mm}{\centering ИТ-системы, Сущности}} & {\bf \parbox[c]{100mm}{\centering Сущности}} \\
\hline
\gofro & Программная система планирования производства Гофротара, Opti-Corrugated\\
 \\
 \hline
\myobject{Контрагенты}  & Справочник предназначен для хранения списка контрагентов. Контрагенты – это поставщики и покупатели, организации и частные лица.
 \\
\hline
\myobject{Номенклатура} & Справочник предназначен для хранения информации о товарах, комплектах, наборах, готовой продукции, возвратной таре, полуфабрикатах, материалах, услугах, оборудовании.
\\
\hline
\myobject{ТехнологическаяКарта} & Справочник «Технологические карты» содержит информацию о технологических картах на изготовление готовой продукции. 
\\
\hline
\myobject{Заявка} & Документ «Заявка» (Заказ покупателя) предназначен для оформления предварительной договорённости с покупателем о намерении приобрести готовую продукцию под свои требования.
\\
\hline
\myobject{Заказ} & Документ «Заказ» (Заказ производства) предназначен для оформления требования для производства заданной продукции в указанном объеме к определенному сроку.
% \\
% \myobject{РаспоряжениеНаОтгрузку} & документ, предназначенный для выдачи распоряжения на отгрузку готовой продукции от менеджера складским службам
% \\
% \myobject{ОприходованиеТМЦ} & складской документ для регистрации факта поступления готовой продукции на склад.
% \\
% \myobject{ПоступлениеТМЦ} & Документ «Поступление ТМЦ» предназначен для отображения движения товарно-материальных ценностей в производстве и регистрации факта поступления материальных запасов, участвующих в производственном цикле.
% \\
% \myobject{ПеремещениеТМЦ} & Документ «Перемещение ТМЦ» предназначен для оформления операции перемещения материалов, товаров, продукции, полуфабрикатов между складами.
% \\
% \myobject{РеализацияТМЦ} & документ «Реализация ТМЦ» предназначен для отображения движения товарно-материальных ценностей в производстве и регистрации факта реализации готовой продукции заказчикам.
% \\
% \myobject{ПриемкаПолуфабрикатов} &документ «Приемка полуфабрикатов» предназначен для позаказного учета объема полуфабрикатов, который не требуется выпускать на гофроагрегате (например при использовании покупного картона).
% \\
% \myobject{СписаниеТМЦ} & складской документ для регистрации факта списания ТМЦ со склада.
\\
\hline
\myobject{ВыработкаПоПереработке}  & Производственный документ, рабочее место машиниста линии переработки. Рабочее место служит для получения плана работы линии переработки, формирования необходимых печатных форм и регистрации факта выработки готовой продукции и полуфабрикатов.
\\
\hline
\myobject{ВыработкаГофроагрегата} & Производственный документ, рабочее место машиниста гофроагрегата. Рабочее место служит для получения плана работы гофроагрегата (раскроев гофрополотна), формирования необходимых печатных форм и регистрации факта выработки полуфабрикатов (заготовок) и готовой продукции (гофролистов).
\\
\hline
\myobject{ЗаявкаНаИзготовлениеШтанц\-Формы} & Документ «Заявка на изготовление штанц-формы» служит основанием для создания или модификации оснастки (штанц-формы). В документе содержится подробная информация о штанц-форме. Документ доступен в том случае, если в настройках системы не установлена галочка в поле «Использовать упрощенный ввод оснастки».
\\
\hline
\myobject{ЗаявкаНаИзготовлениеФлексо\-Формы} & Документ «Заявка на изготовление ФПФ» служит основанием для создания или модификации оснастки (фотополимерной формы). В документе содержится подробная информация о ФПФ. Документ доступен в том случае, если в настройках системы не установлена галочка в поле <<Использовать упрощенный ввод оснастки>>.
\\
\hline
\myobject{Оснастка} & Справочник «Оснастка» содержит информацию по применяемым в ходе производства гофроупаковки фотополимерным печатным формам (ФПФ) и штанц формам. 
\\
\hline
\myobject{Организация}  & Справочник предназначен для хранения списка юридических лиц, входящих в состав предприятия (группы), а также хранения постоянных сведений о них. В этом же справочнике хранятся и сведения об индивидуальных предпринимателях, учет по которым ведется в программе. 
% \\
% \hline
% \myobject{ПретензииКонтрагентов} & Документ предназначен для регистрации поступающих от клиентов претензий по качеству продукции.
\\
\hline
\myobject{ЗаказПоставщику} & Документ «Заказ поставщику» предназначен для фиксации плана потребностей в товарно-материальных ценностях на складах и регистрации факта планирования потребности материальных запасов, участвующих в производственном цикле. Документ содержит информацию о видах ТМЦ, объемах планируемых потребностей для складов.
% \\
% \hline
% \myobject{ОбъемыРабочихЦентров} & Регистр сведений для указания ежедневных доступных мощностей по каждому рабочему центру (используется для предварительного планирования) 
\\
\hline
\myobject{СырьеДляВыработки} & Документ «Сырье для выработки» предназначен для регистрации использования сырья (рулоны бумаги и картона) на раскатах гофроагрегатов.
\\
\hline
\myobject{ВводОстатков} & Документ «ВводОстатков» предназначен для ввода остатков товаров на складах.
% \\
% \hline
% \myobject{ФормированиеПаллет} & Документ «ФормированиеПаллет» предназначен для генерации уникальных кодов паллет на готовую продукцию.
\\

\hline
\caption{Система \blue{\gofro}}\label{bp:system3}
\end{longtable}  
\normalsize





\begin{longtable}{|p{69mm}|p{100mm}|}
\hline
{\bf \parbox[c][5mm]{69mm}{\centering ИТ-системы}} & {\bf \parbox[c]{100mm}{\centering Сущности}} \\
\hline
\blue{@MSExcel} & Программа для работы с электронными таблицами, созданная корпорацией Microsoft для Microsoft Windows.\\
 \hline
\blue{@MSWord} & Программа для работы с электронными текстовыми документами, созданная корпорацией Microsoft для Microsoft Windows.\\
 \hline
%  \blue{@AutoCAD} & Программное обеспечение автоматизированного проектирования (САПР), для создания точных 2D - и 3D-чертежей изделий.\\
%  \hline
% \blue{@CorelDraw} & Графический редактор, предоставляющий собой полный набор инструментов для создания векторной графики.\\
%  \hline
\blue{@SYNCRO} & Система смены заказа на гофроагрегате Fosber\\
\hline
% \blue{@AbobeIllustrator} &	Векторный графический редактор, разработанный и распространяемый компанией Adobe Systems.\\
% \hline
% \blue{@ArtiosCAD} &	Векторный графический редактор для проектирования и разработки упаковки из картона.\\
%  \hline
\caption{Прочие системы}\label{bp:system4}
\end{longtable}  
\normalsize


