\ifx \notincludehead\undefined
    \newcommand*{\No}{\textnumero}
 
    \documentclass[russian, utf8, 12pt, pointsubsection,floatsubsection]{eskdtext}
    \usepackage[russian]{babel}

	\ifx\pdfoutput\undefined  
		\def\pdfoutput{0}
	\fi

	\ifnum\pdfoutput=0 
		\sloppy
		%	\usepackage[dvips]{graphicx}       % загрузка графики под dvi  
		\textheight=250mm                  % для DVI высота печатного текста
		\textwidth=165mm                   % ширина печатного текста
	\else
		%	\usepackage[pdftex]{graphicx}      % загрузка графики под pdf
		\usepackage{cmap}                  % чтоб работал поиск по PDF 
		% гиперссылки в PDF
		\usepackage[unicode, pdftex, colorlinks, linkcolor=blue]{hyperref} 
		\pdfcompresslevel=9                % сжимаем PDF 
		%	\textheight=240mm                  % для PDF высота печатного текста
		%	\textwidth=165mm                   % ширина печатного текста
	\fi 

	\usepackage{eskdchngsheet}
	\usepackage[T2A]{fontenc}
	%\usepackage[cp1251]{inputenc}
	\usepackage{amstext}
	\usepackage{amsmath}
	\usepackage{listings}
	\usepackage{rotating}
	\usepackage{pmasc}
    % пакет placeins позволяет вставлять плавающие объекты (рисунки) в том месте, 
    % где это необходимо. Для вывода рисунка после него встаить команду \FloatBarrier
	\usepackage{placeins}    
    
	\usepackage{array}
	\usepackage{longtable}   % подключение длинных таблиц
	\usepackage{indentfirst} % идентификация первых абзацев после секционирования
	\usepackage{fancyhdr}    % расширенный формат страниц
	\usepackage{ulem}        % подчеркивания текста \uline\uuline\uwave\sout \xout
	%\voffset=-25mm   % -25         % сдвиг страницы вверх
	%\hoffset=-15mm   % -10         % сдвиг страницы влево
    \usepackage{graphicx}
    \usepackage{floatflt} % для рисунков
	\usepackage{wrapfig}  % для рисунков
    \usepackage{todonotes} % тудушки
	\usepackage{xspace}    % убрать лишние пробелы
	\usepackage{comment}   % комментировать блоками
    %\usepackage[usenames]{color}
    %\usepackage{colortbl}

	\sloppy                             % подавление дополнительных переносов
	\righthyphenmin=2                   % можно переносить
	
	\ESKDclassCode{ТП}
	\ESKDtitle{Программная система планирования производства ПС ПП «Opti-Paper»}

	\usepackage{lscape}

	% для Первой спецификации
	\ESKDdocName{Описание бизнес-процессов}
	\ESKDsignature{\ESKDNUM}
	\ESKDcolumnII{\ESKDNUM}
	\ESKDcolumnI{Программная документация. БП. Ред.1}

	\ESKDgroup{ООО <<Опти-Софт>>}
	\ESKDauthor{Косицын Д.П.}
	\ESKDtitleAgreedBy{Директор ООО <<Опти-Софт>>}{Шабаев А.И.}
	\ESKDtitleDesignedBy{Зам. директора <<Опти-Софт>>}{Косицын Д.П.}
    \ESKDtitleDesignedBy{Консультант}{Жернаков Р.В.}	
    \ESKDtitleDesignedBy{Начальник отдела разработки}{Сошкин Р.В.}


	\ESKDtitleApprovedBy{Генеральный директор \FIRMA}{\DIRECTOR}
	%\ESKDtitleApprovedBy{\rule{72pt}{1pt}}{\rule{72pt}{1pt}}
	\ESKDtitleAgreedBy{Директор по производству}{ФИО}
	\ESKDtitleAgreedBy{Коммерческий директор}{ФИО}

	%\ESKDtitleAgreedBy{\rule{72pt}{1pt}}{\rule{72pt}{1pt}}

	\ESKDdate{2024/11/20}
    \frenchspacing
    %	\newcommand*{\No}{\textnumero}
	% для нумерации в длинных enumerate: a, b,...y, z, aa,ab,..
	%\usepackage{alphalph}  
	%\renewcommand{\theenumi}{\alphalph{\value{enumi}}}
\fi
